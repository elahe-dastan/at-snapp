\documentclass[a4]{report}

\usepackage[localise]{xepersian}

\settextfont{Vazir}

\begin{document}

\فصل{گزارش‌های هفتگی}

\قسمت{هفته اول}

\پاراگراف{}
هفته‌ی اول کارآموزی بیشتر صرف شناخت ساختار شرکت و گرفتن دسترسی‌ها و خواندن کدهای قبلی می‌شود.

\پاراگراف{}
ساختار شرکت اسنپ از ساختار اسپاتیفای\پانویس{Spotify} الهام گرفته شده و حالت شطرنجی دارد.
بدین صورت که هر ونچر (منظور از ونچر بیزینس‌های مختلف می‌باشد،
به طور مثال اسنپ کب\پانویس{Snapp Cab} و اسنپ باکس\پانویس{Snapp Box} و اسنپ دکتر\پانویس{Snapp Doctor} سه ونچر متفاوت اند)
یک مدیر فنی ارشد دارد سپس هر ونچر به چندین چپتر\پانویس{Chapter} تقسیم می‌شود برای مثال در ونچر اسنپ کب که من در آن مشغول هستم
دو چپتر برای بک‌اند داریم که نام یک چپتر، آلفا\پانویس{Alpha} است که عمده سرویس‌هایشان با زبان \متن‌لاتین{PHP} توسعه یافته است و روی کد قدیمی و اولیه‌ی اسنپ کار می‌کنند.
چپتر دیگر براوو\پانویس{Bravo} است که با زبان گولنگ و روی میکروسرویس‌های جدید شرکت که از کد قدیمی جدا شده‌اند، کار می‌کند.

\پاراگراف{}
زبان گولنگ کامپایل شده، سرعت خوبی داشته و یادگیری آن ساده است. همگی این دلایل باعث شده‌اند که سرویس‌های جدید شرکت با این زبان توسعه پیدا کنند.

\پاراگراف{}
هر چپتر یک مدیر دارد که وظیفه‌ی سامان‌دهی به آن چپتر هم از نظر مدیریت افراد و هم فنی را دارد.
هر چپتر به ورتیکال‌های مختلف تفسیم می‌شود که هر ورتیکال مسئول توسعه‌ی میکروسرویس‌های مشخصی است.
هر ورتیکال نیز مدیر خود را دارد، در چپتر ما ورتیکال‌های مختلفی مانند شرد سرویسز، دیسپچیگ و … وجود دارد.

\پاراگراف{}
ورتیکالی که من در آن مشغول هستم یوزر نام دارد و تنها ورتیکالی است که به داده‌های خام کاربران دسترسی دارد و وظیفه ی ارتباط سرویس‌های داخلی
با سرویس‌های خارجی دارد مثلا ساخت اکانت و لاگین کاربران اسنپ در دست این ورتیکال است
یا ثبت نام دیجیتال راننده‌ها توسط این ورتیکال صورت میگیرد که اهمیت آن در دوران کرونا دو چندان شده است.
همچنین پنهان کردن شماره های کاربران هنگام تماس با راننده و چندین سرویس دیگر برعهده این ورتیکال است.

\پاراگراف{}
در شرکت با توجه به نیاز ورتیکال‌ها هر ورتیکال می‌تواند تعداد مختلفی نیرو در کنار نیروهای بک‌اند داشته باشد.
در ورتیکال یوزر به علت حساسیت و تنوع سرویس‌ها در کنار نیروهای بک‌اند،
نیروهای فرانت‌اند، اندروید، تست و دواپس\پانویس{DevOps} همگی حضور دارند.
با حضور نیروهای مدیریت محصول و اسکرام مستر\پانویس{Scrum} بزرگ‌ترین ورتیکال شرکت را تشکیل می‌دهد.

\پاراگراف{}
در شرکت یک نربان پیشرفت وجود دارد که هنگام ورود هر کس سطح او به او گفته می‌شود و هر ۶ ماه با توجه به عملکرد فرد که توسط مدیر ورتیکال، مدیر چپتر
و نظر سایر هم‌تیمی‌های او سنجیده می‌شود می‌تواند سطح وی ارتقا پیدا کند.

\پاراگراف{}
در شرکت از متدلوژی اسکرام استفاده می‌شود.
هر فصل در سال یک کوآرتر محسوب می‌شود که در ابتدای آن شرکت اهداف کلی را تعیین می‌کند،
که این اهداف با نظارت مستقیم مدیر فنی ارشد ونچر مشخص می‌شود.
این اهداف در اختیار ورتیکال‌ها قرار می‌گیرد و هر ورتیکال دو هفته فرصت دارد تا اهداف خود را مشخص کند به طوری که در راستای اهداف شرکت قرار بگیرد.
در ابتدای هر کوآرتر جلسات زیادی برای طراحی فنی پروژه‌ها، تخمین زمانی هر یک و \نقاط‌خ می‌شود.
هر کوآرتر به ۶ اسپیرینت\پانویس{Sprint} دو هفته‌ای تقسیم می‌شود و در ابتدای هر اسپیرینت نیز جلساتی برای انتخاب تسک‌ها و تخمین زمان آن ها تشکیل می‌شود.
در انتهای هر اسپیرینت نموداری توسط اسکرام مستر کشیده می‌شود که نشان دهنده‌ی عملکرد اعضا و عملکرد کلی تیم است
و با توجه به آن ظرفیت تیم برای اسپرینت بعدی تخمین زده می‌شود.
همچنین در انتهای هر کوآرتر اسکرام مستر با تک تک اعضا جلساتی خواهد داشت تا از دغدغه‌ها و مشکلات آن‌ها مطلع شود.
یک جلسه رترو\پانویس{Retro} نیز تشکیل خواهد شد تا هر کس نقد ها و پیشنهادات خود را برای کوآرتر بعد بیان کند و
همچنین هر کس باید به صورت ناشناس درباره ی هم تیمی هایش نظر بدهد.

\پاراگراف{}
پس از اینکه یک عضو جدید به شرکت اضافه می‌شود یک فرد از خود تیم به عنوان بادی به او معرفی می شود
که وظیفه دارد او را با تیم و سرویس‌ها آشنا کند و همچنین دسترسی‌های او را برایش فراهم کند.
هر برنامه‌نویس پس از اضافه شدن به تیم باید یک وی پی ان از شرکت بگیرد که تنها به وسیله‌ی آن می‌تواند به سرویس‌ها دسترسی داشته باشد.
همچنین باید دسترسی به ایمیل سازمانی، گیت‌لب\پانویس{Gitlab} که مخزن اصلی نگهداری کدها می‌باشد،
جیرا\پانویس{Jira} که برای مدیریت تسک‌ها می‌باشد،
کانفلوئنس\پانویس{Confluence} که برای مدیریت مستندات می‌باشد
و اسنپ کلاد\پانویس{SnappCloud} که سرویس ابری شرکت مبتنی بر \متن‌لاتین{Openshift} می‌باشد
را بگیرد.
همچنین برای آشنا شدن فرد جدید با تیم در هفته‌ی اول سعی می‌شود جلسات غیرکاری تشکیل شود که در آن اعضا بازی‌های گروهی انجام می‌دهند
و با یکدیگر آشنا می‌شوند.
همچنین فرد جدید در هفته‌ی اول باید ساختار کد‌های شرکت را مطالعه کند
و همچنین مستندات سرویس‌های ورتیکالش را مطالعه کند
تا با معماری سرویس‌های ورتیکال و شرکت آشنا شود
و بداند هر کدام از سوریس‌های تیم با چه سرویس‌هایی از سایر ورتیکال‌ها در ارتباط است.

\قسمت{هفته دوم}

\پاراگراف{}
پس از اینکه با سرویس‌های شرکت به طور کلی آشنا شدم حال باید روی یک سرویس شروع به توسعه می‌کردم.
سرویسی که برای شروع انتخاب شد سرویس ستار بود که عمل پنهان کردن شماره‌های کاربران و رانندگان را انجام می‌دهد تا امنیت بیشتری برای سفر فراهم کند.
به علت قرارداد عدم افشای اطلاعات\پانویس{NDA} که با شرکت امضا شده است و حساسیت کار این ورتیکال از توضیح معماری سرویس یا کد به هر شکل و حتی ذکر نام شرکت‌های
طرف قرارداد کاملا معذورم و به اجبار به توضیحات زیر بسنده می‌کنم.

\پاراگراف{}
سرویس پنهان کردن شماره‌های تماس در شرکت یکی از پایدارترین سرویس‌های شرکت است
که برای انجام کار خود به اطلاعات راننده‌ها و مسافران نیاز دارد
به همین جهت با پایگاه‌های داده‌ای متعددی سر و کار دارد.
همچنین برای سرعت عمل بالا از \متن‌لاتین{Redis} به عنوان حافظه نهان استفاده شده است.
برای پنهان کردن شماره‌ها شرکت با فراهم کنندگان مختلفی قرارداد دارد که ما را در این امر برقراری تماس را برعهده دارند.
سرویس‌های ما نیاز به فراخوانی این فراهم‌کنندگان دارند.
همچنین ستار خود توسط سرویس‌های دیگری از ورتیکال‌های دیگر نیز فراخوانی می‌شود و از این رو سرویس ستار
کیت توسعه سرویس (SDK) نیز دارد.
به علت وابستگی بالای این سرویس به پایگاه‌های داده‌ای مختلف و علم به اینکه پایگاه‌های داده‌ای
برای زیرساخت‌های ابری مناسب نیستند، تا اکنون این سرویس روی ماشین‌های مجازی بوده که یکی از تسک‌های مهم آن بردن این سرویس بر روی ابر است.
همچنین این سرویس تا کنون برای حالت اسنپ برای دیگری فعال نبوده که یکی از تسک‌های مهم اضافه کردن این ویژگی به آن است که طبیعتا باعث
می‌شود نیاز داشته باشیم با سرویس در شرکت به نام مورفیوس که وظیفه‌ی ارسال پیامک را دارد جهت اطلاع‌رسانی شماره‌ی پنهان شده به کاربر نیز در ارتباط داشته باشیم.
اولین مشکلی که در در این سرویس وجود داشت بحث نشت حافظه آن بود.
زمانی که در سیستم مانیتورینگ به عملکرد این سرویس در یک بازه‌ی زمانی نستا بزرگ نگاه می‌کردیم قابل مشاهده بود که حافظه مصرفی این سرویس با شیب کمی همواره در حال
افزایش است.
البته این حل این مشکل اولویت شرکت نبوده چرا که اولا اگر این سرویس از کار بیافتد نهایتا شماره‌ها پنهان نخواهند شد
که این امر در عملکرد کلی شرکت خللی وارد نمی‌کند و از طرفی چندین نسخه از این کد بالا آورده شده است که هر زمان مموری مصرفی هر کدام از حد مشخصی عبور کرد
آن نسخه با توجه به تنظیمات صورت گرفته به طور خودکار راه‌اندازی مجدد خواهد شد.
تسک اول من جستجو در کد و تحقیق جهت پیدا کردن مشکل نشتی حافظه و حل آن بود.
در جهت جستجو برای حل این مشکل اقدامات زیر انجام شد.

\end{document}
